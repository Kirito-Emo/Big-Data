\chapter{Esercizio 1 -- Hadoop MapReduce}
	L’obiettivo del primo esercizio era realizzare un programma MapReduce che analizzasse i dati del dataset \texttt{BMW\_Car\_Sales\_Classification.csv} applicando almeno tre trasformazioni.
	Per affrontare il problema è stata progettata una \textbf{pipeline composta da tre job sequenziali}, orchestrati dal driver \texttt{DriverBMWSales}.
	Ogni job prende in input l’output del precedente, con l’ultimo passo che produce il risultato finale richiesto.
	
	\section{Job 1: Aggregazione per regione e modello}
		Il primo job ha lo scopo di aggregare i dati di vendita per \textbf{coppia (regione, modello)}.
		
		\begin{itemize}
			\item \textbf{Mapper1:} legge ciascuna riga del file CSV (ignorando l’intestazione) ed emette come chiave la coppia \texttt{region-model}. \\
			Il valore associato è una tupla codificata nel formato \texttt{1|volume|prezzo|isHigh}, dove:
			\begin{itemize}
				\item \texttt{1} rappresenta il conteggio di un record;
				\item \texttt{volume} è il numero di unità vendute;
				\item \texttt{price} è il prezzo della singola vendita;
				\item \texttt{isHigh} vale 1 se la classificazione delle vendite è \texttt{High}, altrimenti 0.
			\end{itemize}
			
			\item \textbf{Combiner1} e \textbf{Reducer1:} sommano i valori parziali generati dal mapper, producendo per ogni chiave (regione, modello) l’aggregato finale nel formato \texttt{count|sumVolume|sumPrice|highCount}.
		\end{itemize}
		
		L’output di questo job è, quindi, un insieme di statistiche aggregate per modello e regione.
	
	\section{Job 2: Calcolo dei totali per regione}
		Il secondo job utilizza l’output del Job 1 per calcolare i \textbf{totali complessivi delle vendite per ogni regione}.
		
		\begin{itemize}
			\item \textbf{Mapper2:} estrae da ciascuna riga la regione e il valore \texttt{sumVolume};
			\item \textbf{Reducer2:} somma tutti i valori per regione e produce come output \texttt{region - regionTotalVolume}.
		\end{itemize}
		
		Questo passo fornisce i volumi totali regionali necessari per calcolare le percentuali di mercato.
	
	\section{Job 3: Top-K modelli per regione}
		Il terzo job combina i risultati dei due precedenti per ottenere statistiche avanzate e selezionare i \textbf{Top-K modelli per regione}.
		
		\begin{itemize}
			\item \textbf{Mapper3:} legge i dati del Job 1 e, tramite i totali regionali (caricati dall’output del Job 2), calcola per ogni modello:
			\begin{itemize}
				\item la quota percentuale di mercato (\texttt{share\%});
				\item il prezzo medio (\texttt{avgPrice});
				\item la percentuale di classificazioni alte (\texttt{highShare}).
			\end{itemize}
			
			L’output del mapper ha come chiave la regione e come valore le metriche associate al modello.
			
			\item \textbf{Reducer3:} ordina i modelli di ciascuna regione in base al volume di vendite e ne seleziona i primi \texttt{K} (default 5). \\
			L’output finale ha la forma:
			\[
				\texttt{region \quad model \quad sumVol \quad share\% \quad avgPrice \quad highShare}
			\]
		\end{itemize}
	
	\section{Sintesi della pipeline}
		La catena di tre job realizza quindi il seguente flusso di trasformazioni:
		
		\begin{enumerate}
			\item aggregazione per \texttt{(regione, modello)};
			\item calcolo dei totali di vendita per regione;
			\item calcolo delle metriche per modello e selezione dei Top-K.
		\end{enumerate}
		
		Questa architettura permette di ottenere statistiche dettagliate sulle vendite BMW, evidenziando i modelli più rilevanti per ciascuna regione, sia in termini di quota di mercato che di prezzo medio.